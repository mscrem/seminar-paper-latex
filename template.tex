% !TEX TS-program = xelatex
\documentclass[12pt]{article}

% Sprache english, ngerman
\usepackage[ngerman]{babel}

% Schriftart
\usepackage{fontspec}
\setmainfont{Times New Roman}

% Zitate
\usepackage[
  backend=bibtex,
  style=authoryear
]{biblatex}
\addbibresource{PATH/references.bib} % .bib datei verlinken
% für Kurzzitate, wie Wittgenstein oder Aristoteles
\DeclareCiteCommand{\shortcite}[\mkbibparens]
  {\usebibmacro{prenote}}
  {\printfield{shorttitle}}      % nur shorttitle
  {\multicitedelim}
  {\addspace\printfield{postnote}}
\DeclareFieldFormat{postnote}{#1} %entfernt p. vor Seitenzahl

% Seitenlayout
\usepackage[a4paper, left=2.5cm, right=2.5cm, top=2.5cm, bottom=2cm]{geometry}

% Zeilenabstand 1,5
\usepackage{setspace}
\setstretch{1.5}

% Absatzabstand statt Einzug
\setlength{\parindent}{0pt}
\setlength{\parskip}{0.1\baselineskip}

% Nummerierte Überschriften wie in Word
\usepackage{titlesec}
\titleformat{\section}
  {\normalfont\Large\bfseries}{\thesection.}{1em}{}
\titleformat{\subsection}
  {\normalfont\large\bfseries}{\thesubsection.}{1em}{}
\titleformat{\subsubsection}
  {\normalfont\normalsize\bfseries}{\thesubsubsection.}{1em}{}

% Inhaltsverzeichnis ab Seite 2, Seitenzahlen ab Seite 3
\usepackage{fancyhdr}
\usepackage{afterpage}
\usepackage{tocloft}
\renewcommand{\cftsecleader}{\cftdotfill{\cftdotsep}}
\renewcommand{\cftsubsecleader}{\cftdotfill{\cftdotsep}}
\renewcommand{\cftsubsubsecleader}{\cftdotfill{\cftdotsep}}

% Seitenzahlen erst ab Seite 3 (Seite 1 und 2 ohne Nummer)
\pagenumbering{gobble}

\begin{document}

% Titelseite (Seite 1)
\begin{titlepage}
  \centering
  \vspace*{5cm}

  {\Large Seminararbeit \par}
  {\Huge\bfseries Titel \par}
  \vspace{2cm}

  {\Large LV-Titel \par}
  {\Large Prof  \par}
  {\Large Semester \par}

  \vfill

  {\Large Name \par}
  {\Large Matrikelnummer \par}
\end{titlepage}

% Inhaltsverzeichnis (Seite 2)
\tableofcontents
\clearpage

% Seitenzahlen ab hier beginnen
\pagenumbering{arabic}
\setcounter{page}{1}

% Hier startet die eigentliche Arbeit
% normales Zitat:  \parencite[Seite(n)]{bibtexeintrag}
% shortform, z.B. Aristoteles: \shortcite[Stelle]{bibtexeintrag}
\section{Einleitung}

\clearpage


\section{Überschrift}

\subsection{Unterüberschrift}

\clearpage


\section{Schluss}

\clearpage

% Literaturverzeichnis
\printbibliography[heading=bibintoc]
\end{document}

